% !TEX program = xelatex
\documentclass{matthijs}

\usepackage{wallpaper}
\usepackage{color}

% Versioning
\usepackage[maxdepth=2]{gitinfo2}

% Long text column, automatically wraps to next line
\newcolumntype{R}{>{\arraybackslash}m{10cm}}

\begin{document}

	\taal{en}

	\begin{titelpagina}
		\color{white}

		\titel{\vspace{50pt}Timetable}{\emph{Project Lane Detection using FPGAs}}
		\author{
			\begin{tabular}{r l}
				\textbf{Author:} & Matthijs Bakker{\color{white}\footnote{\color{white}\textsuperscript{1} s1142121@student.windesheim.nl}} \\
				\textbf{Course:} & HBO-ICT ESA Full-Time \\
				\\
				\textbf{Company:} & AROBS Transilvania SA, Cluj-Napoca, Romania \\
				\textbf{Company Supervisor:} & Pangyu Jeong \\
				\textbf{Windesheim Supervisor:} & Willie Conen \\
				\\
				\textbf{Version:} & 1.0 \\
				\textbf{Commit:} & \gitAbbrevHash @master \\
			\end{tabular}
			\vspace{4cm}
		}

		\ThisULCornerWallPaper{1.001}{asset_bg_first_page.jpg}
	\end{titelpagina}

	\begin{hoofdstuk}{Timetable}

		In this document, I kept track of how many hours I worked per day.
		At the end of the project I should have put around 840 hours into the project (40 hours per week for a period of 21 weeks.)
		
		\bigskip

		AROBS works with a time management tool which tracks how long you have been in the office each day.
		I have partially based the hours in the table below on the data that was available in this time tracking tool.
		
		\bigskip

		For various reasons, I occasionally worked from my apartment.
		This was not a problem, because AROBS allows me to work remotely.
		I can take the work laptop back to my apartment and work on the documentation and such.

		\bigskip

		% TODO use calctab or similar to automatically calculate
		% the cumulative hours statistic.
		\begin{tabel}{Timetable}{l @{\extracolsep{\fill}} c c R}
			\emph{Date} & \emph{Hours} & \emph{Cm.Hrs} & \emph{Activity} \\
			\midrule

			31-1-2022 & 2 & 2 & Create this document and the project plan \tabularnewline
			1-2-2022 & 3 & 5 & Start writing project plan and read stage guide \tabularnewline
			2-2-2022 & 8 & 13 & Meeting each other at AROBS building \tabularnewline
			3-2-2022 & 8 & 21 & Finish concept version of the Plan of Action \tabularnewline
			4-2-2022 & 8 & 29 & Integrate feedback for the PoA and submit it \tabularnewline
			7-2-2022 & 9 & 38 & Start writing the image file handling support for the computer program \tabularnewline
			8-2-2022 & 8 & 46 & Internship meeting + finish image file handling support for the computer program \tabularnewline
			9-2-2022 & 8 & 54 & Start writing the research paper + integrate Actions workflow for documents \tabularnewline
			10-2-2022 & 8 & 62 & Worked from home, collected papers and made progress with research \tabularnewline
			11-2-2022 & 7 & 69 & Worked from home, researching line classification techniques \tabularnewline
			14-2-2022 & 10 & 79 & Implemented Gaussian and Sobel filters into the application \tabularnewline
			15-2-2022 & 8 & 87 & Progress with research paper \tabularnewline
			16-2-2022 & 8 & 95 & Rearrange paper and progress \tabularnewline
			17-2-2022 & 7 & 101 & Try to implement Hough \tabularnewline
			18-2-2022 & 10 & 111 & Finish implementing Hough + fix a lot of issues \tabularnewline
			21-2-2022 & 8 & 119 & Make lots of progress with the paper \tabularnewline
			22-2-2022 & 9 & 128 & Same story as yesterday \tabularnewline
			23-2-2022 & 9 & 136 & K-means implementation and code cleanup \tabularnewline 
			24-2-2022 & 8 & 144 & Write about Sobel, Laplacian, etc. \tabularnewline
		\end{tabel}

	\end{hoofdstuk}

	% Empty last page
	\clearpage
	\thispagestyle{empty}
	\addtocounter{page}{-1}
	\ThisULCornerWallPaper{1.005}{asset_bg_last_page.jpg}
	\
	\clearpage

\end{document}
